\documentclass[12pt]{article}
\usepackage[numbers]{natbib}
\usepackage[margin=1in]{geometry}
\usepackage[nottoc]{tocbibind}
\usepackage{setspace}

\doublespacing

\author{Thomas Kwashnak}
\title{Senior Thesis}

\newcommand{\CC}{C\nolinebreak\hspace{-.05em}\raisebox{.4ex}{\tiny\bf +}\nolinebreak\hspace{-.10em}\raisebox{.4ex}{\tiny\bf +}}

\begin{document}

\maketitle

\newpage

\newpage

\section{Abstract}

\section{Introduction}


\paragraph{}
Even in popular machine learning libraries, there are often times shortcuts or bugs that developers have not yet gotten to \cite{article_deep_learning_framework_debt}.


\section{Methods}

\paragraph{} 
To test the performance difference between manual implementations and the TensorFlow library, an identical algorithm was implemented in multiple different frameworks and languages.
This includes CUDA \cite{lib_cuda}, which runs on the GPU, TensorFlow \cite{lib_tensorflow}, which runs on either the GPU or CPU, and Rust \cite{lang_rust}, which acts as the CPU implementation.
While identical algorithms were not feasable to implement during the alloted time frame, each algorithm attempts to be as fast is it can be for the particular framework.
The algorithm chosen is a simple back-propagation neural network.
This allows for experimenting with the number of variables and size of the network, as well as the number of observations being concurrently processed at once.

To make each process fair, the algorithms are fed the exact same set of values and data.
Initially, the algorithms will build the neural network using pre-randomized weights and biases.
Then, a bootstrapped set of observations is fed into the network, along with the label to train off of.
This process repeats for various combinations of variable counts, or the number of variables in the input vector, and observation counts, or the number of concurrent observations being proessed at one time.

\paragraph{Rust}
The Rust \cite{lang_rust} implementation acts as the manual CPU implementation.
It uses the Rust standard libraries along with the Rayon crate, which provides APIs to simplify multi-threading Rust appl


\paragraph{CUDA}
Using \CC  \cite{lang_c++} and CUDA \cite{lib_cuda}.

\paragraph{TensorFlow}
Using Python \cite{lang_python} and TensorFlow \cite{lib_tensorflow}.

\section{Results}

\section{Discussion}


\section{Conclusion}

\newpage
\bibliographystyle{ieeetr}
\bibliography{refs}

\end{document}
